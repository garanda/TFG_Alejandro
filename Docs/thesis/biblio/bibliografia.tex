\begin{thebibliography}{99}

\addcontentsline{toc}{chapter}{Bibliografía.}

\bibitem{alonso} Alonso-Jiménez, J.A., Aranda-Corral, G.A., Borrego-Díaz, J., Fernández-Lebrón,
M.M., Hidalgo-Doblado, M.J. (2008). Extending Attribute Exploration by Means of Boolean Derivatives. In: Proc. 6th Int. Conf. on Concept Lattices and Their Applications. CEUR Workshops Proc., p. 433. 

\bibitem{algoritmo} Aranda-Corral, G.A., Borrego-Díaz, J. (2010). Reconciling Knowledge in Social Tagging Web Services. Lecture Notes in Computer Science, 6077(1), 101-121.

\bibitem{mowento} Aranda-Corral, G.A., Borrego-Díaz, J., Gómez-Marín, F. (2009). Toward Semantic Mobile Web 2.0 through Multiagent Systems. In: Håkansson, A., Nguyen, N.T., Hartung, 
R.L., Howlett, R.J., Jain, L.C. (eds.) KES-AMSTA 2009. LNCS, vol. 5559, pp. 400–409. Springer, Heidelberg.

\bibitem{blondel} Blondel, V.D., Guillaume, J.L., Lambiotte, R., Lefebvre, E. (2008). Fast unfolding of communities in large networks. Journal of Statistical Mechanics: Theory and Experiment, Vol. 2008, No. 10.

\bibitem{brooks} Brooks, C.H., Montanez, N. (2006). Improved annotation of the blogosphere via autotagging and hierarchical clustering. In WWW 06. Proceedings of the 15$^{th}$ international conference on World Wide Web, (pp.625-632.). New York: ACM Press.  

\bibitem{afc} Ganter, B., Wille, R. (1999). Formal Concept Analysis - Mathematical Foundations. Springer, Heidelberg.

\bibitem{golder} Golder, S., Huberman, B.A. (2006). The structure of collaborative tagging systems. Journal of Information Science 32(2), 98-208.

\bibitem{gruber} Gruber, T. (2007). Ontology of Folksonomy: A Mash-up of Apples and Oranges. Int'l. Journal on Semantic Web \& Information Systems 3(2).

\bibitem{halpin} Halpin, H., Valentin R., and Hana S. (2006). The dynamics and semantics of collaborative tagging. Proceedings of the 1$^{st}$ Semantic Authoring and Annotation Workshop (SAAW06). 

\bibitem{jaschke} Jäschke, R., Hotho, A., Schmitz, C., Ganter, B., Stumme, G. (2008). Discovering shared conceptualizations in folksonomies. Journal of Web Semantics 6(1), 38–53.

\bibitem{kim} Kim, H.-L., Scerri, S., Brslin, J., Decker, S., Kim, H.-G. (2008). The state of the art in tag ontologies: A semantic model for tagging and folksonomies. In: International Conference on Dublin Core and Metadata Applications, Berlin, Germany.

\bibitem{knerr} Knerr, T. (2006). Tagging ontology-towards a common ontology for folksonomies. \url{http:// tagont.googlecode.com/files/TagOntPaper.pdf}.

\bibitem{mathes} Mathes, A. (2004). Folksonomies - Cooperative Classification and Communication Through Shared Metadata. \url{http://www.adammathes.com/academic/computer-mediated-communication/folksonomies.html}

\bibitem{mika} Mika, P. (2005). Ontologies Are Us: A unified model of social networks and semantics. Proceedings of the 4$^{th}$ International Semantic Web Conference, ISWC 2005, Galway, Ireland, (pp. 522-536). Berlin, Heidelberg: Springer.  

\bibitem{rowley} Rowley, J. (1995). Organizing Knowledge. 2$^{\mbox{\scriptsize nd}}$ Ed. Brookfield, VT: Gower.

\bibitem{shirky} Shirky, C. (2005). Ontology is Overrated: Categories, Links and Tags. \url{http://www.shirky.com/writtings/ontology_overrated.html}

\bibitem{smith} Smith, G. (2007). Tagging: People-Powered Medatada for the Social Web. First. New Riders Publishing, Indianapolis.

\bibitem{tanaka} Tanaka, J.W., Taylor, M. (1991) . Object categories and expertise: Is the basic level in the eye of the beholder? Cognitive Psychology 23(3), 457–482.

\bibitem{van} Van Damme, C., Hepp, M., Siorpaes, K. (2007). FolksOntology: An Integrated Approach for Turning Folksonomies into Ontologies. In: ESWC 2007 workshop Bridging the Gap between Semantic Web and Web 2.0, May 2007, pp. 57–70.

\bibitem{weick} Weick, K.E., Sutcliffe, K.M., Obstfeld, D. (2005). Organizing and the Process of Sense\-making. Organization Science 16(4), 409–421.

\bibitem{yeung} Yeung, C.M.A., Gibbins, N., Shadbolt, N. (2009). Contextualising Tags in Collaborative Tagging Systems. In: Proceedings of the 20th ACM Conference on Hypertext and 
Hypermedia.


\bibitem{conexp} ConExp: \url{http://conexp.sourceforge.net/}

\bibitem{delicious} Delicious: \url{http://www.delicious.com/}

\bibitem{dot} Notación DOT: \url{http://www.graphviz.org/Documentation.php}

\bibitem{fipa} FIPA (Foundation for Intelligent Physical Agents): \url{http://www.fipa.org/}

\bibitem{gephi} Gephi: \url{http://gephi.org/}

\bibitem{jade} JADE (Java Agent Development Framework): \url{http://jade.tilab.com/}

\end{thebibliography}







